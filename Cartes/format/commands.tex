\cardfrontstyle{headings}
\cardbackstyle{plain}
\renewcommand{\familydefault}{\sfdefault} % Ecrire en sans serif par défaut

\pgfkeys{/pgf/number format/precision=2}

\pgfmathparse{\textScale*10}
\setlength{\cardmargin}{\pgfmathresult pt}

\newcommand\pgfmathparseFPU[1]{\begingroup% % Nécessaire pour aller au dela de 16000
\pgfkeys{/pgf/fpu,/pgf/fpu/output format=fixed}%
\pgfmathparse{#1}%
\pgfmathsmuggle\pgfmathresult\endgroup}%


% CREATION DE CARTE
\newcommand{\card}[3]{%
\begin{flashcard}[\scalefont{\textScale}Joule Box]{\includegraphics[height=0.6\linewidth]{#1}\vfill{\scalefont{\textScale}#2}}
  \PowerToBikes{#3}
\end{flashcard}%
}


% FONCTION POWERTOBIKE
\newcommand{\PowerToBikes}[1]
{
\xdef\P{#1}
\xdef\nMax{100}
\pgfmathparseFPU{ceil(\P/\etalon)}
\pgfmathparse{int(\pgfmathresult)}
\xdef\nCyclistes{\pgfmathresult}
\pgfmathparse{int(min(\nCyclistes,\nMax))}
\xdef\nCyclistesDessines{\pgfmathresult}

\pgfmathparseFPU{ceil(sqrt(\nCyclistesDessines))}
\xdef\numLine{\pgfmathresult}
\pgfmathparseFPU{0.8/\numLine}
\xdef\s{\pgfmathresult}

\foreach \n in{1,...,\nCyclistesDessines}{%
	\pgfmathrandominteger{\i}{1}{4}%
	\includegraphics[width=\s\linewidth]{content/cycliste_\i}%
	\pgfmathparse{int(\n < \nCyclistesDessines)}%
	\xdef\r{\pgfmathresult}
	\ifnum\r=1{\hfil }%
	\else%
	\fi%
	}%
	
\vfill

\pgfmathparse{int(\nCyclistes > \nMax)}
\xdef\rbis{\pgfmathresult}
\ifnum\rbis=1
	{\pgfmathparse{int(\nCyclistes-\nMax)}
	\Large\scalefont{\textScale}$\boldsymbol{+\,\pgfmathresult}$}
\else
\fi%

\vfill

% Affichage puissance et nombre de cyclistes
\pgfmathparse{int(\nCyclistes > 1)}
\xdef\rter{\pgfmathresult}
\ifnum\rter=1
	\Large\scalefont{\textScale}$\boldsymbol{\nCyclistes}$ {\bf cyclistes} 
\else
	\Large\scalefont{\textScale}$\boldsymbol{\nCyclistes}$ {\bf cycliste} 
\fi
de $\age$ ans\\
{\large\it\scalefont{\textScale}#1 Watt}
}