\cardfrontstyle{headings}
\cardbackstyle{plain}
\renewcommand{\familydefault}{\sfdefault} % Ecrire en sans serif par défaut

\pgfkeys{/pgf/number format/precision=2}

\pgfmathparse{\textScale*10}
\setlength{\cardmargin}{\pgfmathresult pt}


% CREATION DE CARTE
\newcommand{\card}[3]{%
\begin{flashcard}[\scalefont{\textScale}Joule Box]{\includegraphics[height=0.6\linewidth]{#1}\vfill{\scalefont{\textScale}#2}}
  \PowerToBikes{#3}
\end{flashcard}%
}


% FONCTION POWERTOBIKE
\newcommand{\PowerToBikes}[1]
{
\xdef\P{#1}
\xdef\nMax{100}
\pgfmathparse{int(ceil(\P/\etalon))}
\xdef\nCyclistes{\pgfmathresult}
\pgfmathparse{min(\nCyclistes,\nMax)}
\xdef\nCyclistesDessines{\pgfmathresult}

\pgfmathparse{ceil(sqrt(\nCyclistesDessines))}
\xdef\numLine{\pgfmathresult}
\pgfmathparse{0.8/\numLine}
\xdef\s{\pgfmathresult}

\foreach \n in{1,...,\nCyclistesDessines}{%
	\pgfmathrandominteger{\i}{1}{4}%
	\includegraphics[width=\s\linewidth]{content/cycliste_\i}%
	\pgfmathparse{\n < \nCyclistesDessines}%
	\ifthenelse{\pgfmathresult>0}{ \hfil }{}%
	}%
\pgfmathparse{\nCyclistes > \nMax}
\ifthenelse{\pgfmathresult>0}{\pgfmathparse{int(\nCyclistes-\nMax)}\\[1ex]\Large\scalefont{\textScale}$\boldsymbol{+\,\pgfmathresult}$}{}
\vfill
% Affichage puissance et nombre de cyclistes
{\large\it\scalefont{\textScale}#1 Watt}\\
\pgfmathparse{\nCyclistes > 1}
\ifthenelse{\pgfmathresult>0}{\Large\scalefont{\textScale}$\boldsymbol{\nCyclistes}$ {\bf cyclistes} de $\age$ ans}
{\Large\scalefont{\textScale}$\boldsymbol{\nCyclistes}$ {\bf cycliste} de $\age$ ans}
}